%-------------------------------------------------------------------------------
% SECTION TITLE
%-------------------------------------------------------------------------------
\cvsection{Presentations, Talks}


%-------------------------------------------------------------------------------
% CONTENT
%-------------------------------------------------------------------------------
% \begingroup
% \allowdisplaybreaks
\begin{cvtalks}
%---------------------------------------------------------

\cvtalk
    {UMTC-UMD Postdoc Seminar Program} % Seminar
    {University of Minnesota, Duluth}
    {Understanding Contrastive Learning from Variational and Neural Network Optimization Perspectives} % talk subject
    {2024} % Date(s)
\cvtalk
    {(Poster) SIAM Conference on Mathematics of Data Science (MDS24)} % Seminar
    {Atlanta Georgia}
    {Understanding Contrastive Learning from Variational and Neural Network Optimization Perspectives} % talk subject
    {2024} % Date(s)
\cvtalk
    {SIAM Conference on Mathematics of Data Science (MDS24)} % Seminar
    {Atlanta Georgia}
    { Understanding Contrastive Learning from Variational and Neural Network Optimization Perspectives} % talk subject
    {2024} % Date(s)
\cvtalk
    {SIAM Annual Meeting (SIAM AN24)} % Seminar
    {Spokane, Washington}
    {Geometry-preserving encoder and decoder on latent diffusion models} % talk subject
    {2024} % Date(s)
  \cvtalk
    {SE Machine Learning and Data Science Seminar} % Seminar
    {University of Vienna}
    {Monotone Generative Modeling via a Gromov-Monge Embedding.} % talk subject
    {2024} % Date(s) 4/17
\cvtalk
    {Computational and Applied Math (CAM) Seminar} % Seminar
    {Georgia Tech}
    {Monotone Generative Modeling via a Gromov-Monge Embedding.} % talk subject
    {2024} % Date(s)
\cvtalk
    {MDL Collective Seminar}
    {Iowa State University}
    {Deep JKO: time-implicit particle methods for general nonlinear gradient flows}
    {2024}
\cvtalk
    {Applied Math Seminar} % Seminar
    {UC Santa Barbara}
    {Monotone Generative Modeling via a Gromov-Monge Embedding.} % talk subject
    {2024} % Date(s)
\cvtalk
    {Workshop on Models and Algorithms for Path Planning (MAPP)} % Seminar
    {UT Austin}
    {Monotone Generative Modeling via a Gromov-Monge Embedding.} % talk subject
    {2023} % Date(s)
\cvtalk
    {Analysis and Probability Seminar}
    {Iowa State University}
    {Monotone discretizations of levelset convex geometric PDEs.}
    {2023}
\cvtalk
    {A Monthly Seminar at The Mokaplan Research, Mokameeting} % Seminar
    {Inria Paris}
    {Monotone Generative Modeling via a Gromov-Monge Embedding.} % talk subject
    {2023} % Date(s)
\cvtalk
    {Kernel Club} % Seminar
    {Colorado School of Mines}
    {Monotone Generative Modeling via a Gromov-Monge Embedding.} % talk subject
    {2023} % Date(s)

    % \pagebreak
\cvtalk
    {ACMD Seminar} % Seminar
    {National Institute of Standards and Technology (NIST)}
    {Monotone Generative Modeling via a Gromov-Monge Embedding.} % talk subject
    {2023} % Date(s)
\cvtalk
    {Workshop on kinetic and optimal transport} % Seminar
    {University of Minnesota}
    {Deep JKO for general gradient flows.} % talk subject
    {2023} % Date(s)
  
  \cvtalk
    {2023 Algorithms for Threat Detection PI Workshop (Poster session)} % Seminar
    {Washington D.C.}
    {Monotone generative modeling via Gromov-Monge Embedding.} % talk subject
    {2023} % Date(s)
  \cvtalk
    {The Level Set Collective II} % Seminar
    {UCLA}
    {Monotone discretizations of levelset convex geometric PDEs.} % talk subject
    {2023} % Date(s)
  \cvtalk
    {SIAM Conference on Computational Science and Engineering } % Seminar
    {CSE23}
    {Monotone discretizations of levelset convex geometric PDEs.} % talk subject
    {2023} % Date(s)

  \cvtalk
    {IMA Data Science Seminar} % Seminar
    {University of Minnesota}
    {The back-forth method for Wasserstein gradient flows.} % talk subject
    {2022} % Date(s)

  \cvtalk
    {Optimal transport and Mean field games Seminar} % Seminar
    {University of South Carolina}
    {Mean field control problems for vaccine distribution.} % talk subject
    {2021} % Date(s)
  \cvtalk
    {Current Literature in Applied Mathematics Seminar} % Seminar
    {UCLA}
    {The back-forth method for Wasserstein gradient flows.} % talk subject
    {2021} % Date(s)
  \cvtalk
    {Optimal transport and Mean field games Seminar} % Seminar
    {University of South Carolina}
    {The back-forth method for Wasserstein gradient flows.} % talk subject
    {2020} % Date(s)
  \cvtalk
    {The Level Set Collective} % Seminar
    {UCLA}
    {Numerical Methods and Applications of Optimal Transport.} % talk subject
    {2020} % Date(s)
  \cvtalk
    {Optimal transport and Mean field games Seminar} % Seminar
    {UCLA}
    {Tropical Wasserstein Distances in Phylogenetic Tree Space.} % talk subject
    {2019} % Date(s)
    \cvtalk
    {Optimal transport and Mean field games Seminar} % Seminar
    {UCLA}
    {Energy-efficient Velocity Control for Massive Rotary-Wing UAVs: A Mean Field Game Approach} % talk subject
    {2019} % Date(s)
\end{cvtalks}
% \endgroup

